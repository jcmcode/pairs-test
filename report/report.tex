\documentclass[12pt, a4paper]{article}

% ============================================================
% PACKAGES
% ============================================================
\usepackage[utf8]{inputenc}
\usepackage[T1]{fontenc}
\usepackage{amsmath, amssymb, amsfonts}
\usepackage{graphicx}
\usepackage[margin=1in]{geometry}
\usepackage{hyperref}
\usepackage{booktabs}
\usepackage{float}
\usepackage{caption}
\usepackage{subcaption}
\usepackage{algorithm}
\usepackage{algorithmic}
\usepackage{listings}
\usepackage{xcolor}
\usepackage{natbib}
\usepackage{setspace}
\usepackage{fancyhdr}
\usepackage{enumitem}
\usepackage{tikz}

% ============================================================
% CONFIGURATION
% ============================================================
\hypersetup{
    colorlinks=true,
    linkcolor=blue!70!black,
    citecolor=green!50!black,
    urlcolor=blue!60!black,
}

\lstset{
    basicstyle=\ttfamily\small,
    backgroundcolor=\color{gray!5},
    frame=single,
    rulecolor=\color{gray!30},
    breaklines=true,
    numbers=left,
    numberstyle=\tiny\color{gray},
    keywordstyle=\color{blue!70!black},
    commentstyle=\color{green!50!black},
    stringstyle=\color{red!60!black},
    language=Python,
}

\pagestyle{fancy}
\fancyhf{}
\fancyhead[L]{\small Transient Correlation Detection via Clustering}
\fancyhead[R]{\small Queen's Quant Club}
\fancyfoot[C]{\thepage}

\onehalfspacing

% ============================================================
% TITLE
% ============================================================
\title{
    \vspace{-1cm}
    \textbf{Detecting and Predicting Transient Correlation Formations \\
    Between Assets Using Clustering Algorithms} \\[0.5cm]
    \large OPTICS, K-Means, and DBScan: \\
    A Comparative Study in the Semiconductor Sector
}

\author{
    % TODO: Add all group member names
    Member 1 \and Member 2 \and Member 3 \and Jack Moores \\[0.3cm]
    \normalsize Queen's University Quant Club
}

\date{February 2026}

\begin{document}

\maketitle
\thispagestyle{empty}

\newpage
\tableofcontents
\newpage

% ============================================================
% ABSTRACT
% ============================================================
\begin{abstract}
% TODO: Write abstract (150-250 words)
% Should cover:
% - Problem: detecting transient (event-driven) correlations between assets
% - Approach: OPTICS, K-Means, DBScan clustering on engineered features
% - Universe: semiconductor sector equities
% - Key findings: which method performed best, prediction AUC ranges, tradeable pair statistics
% - Conclusion: practical applicability for pairs trading
\end{abstract}

% ============================================================
% 1. INTRODUCTION
% ============================================================
\section{Introduction}

% TODO: Write introduction (~1-1.5 pages)
% Cover the following points:

\subsection{Motivation}
% - Traditional pairs trading relies on long-term cointegration
% - Many profitable relationships are transient and event-driven
% - News events, earnings, supply chain disruptions create short-lived correlations
% - Need methods that can detect these formations in real time

\subsection{Problem Statement}
% - Given a universe of equities, detect when transient correlation formations occur
% - Validate whether these formations represent tradeable mean-reverting relationships
% - Predict when future formations will occur using historical feature patterns

\subsection{Contributions}
% - Application of three clustering algorithms (OPTICS, K-Means, DBScan) to transient correlation detection
% - Comparative analysis across methods on the same dataset
% - Factor attribution framework identifying which features precede formations
% - Predictive modelling pipeline with time-series cross-validation

\subsection{Report Structure}
% - Brief overview of what each section covers

% ============================================================
% 2. LITERATURE REVIEW
% ============================================================
\section{Literature Review}

% TODO: Write literature review (~1-1.5 pages)

\subsection{Pairs Trading}
% - Gatev et al. (2006) - distance method for pairs trading
% - Vidyamurthy (2004) - cointegration-based pairs trading
% - Limitations of traditional approaches: assume stationarity, miss transient relationships

\subsection{Clustering in Finance}
% - Applications of clustering to asset grouping and regime detection
% - K-Means in portfolio construction
% - Density-based methods for anomaly detection in markets

\subsection{OPTICS and Density-Based Clustering}
% - Ankerst et al. (1999) - OPTICS algorithm
% - Advantages over DBScan for variable-density data
% - Applications in financial time series

\subsection{Event-Driven Correlations}
% - Literature on how news events create transient correlation spikes
% - Semiconductor sector as a case study (supply chain interdependencies)

% ============================================================
% 3. DATA
% ============================================================
\section{Data}

% TODO: Write data section (~1 page)

\subsection{Asset Universe}
% - Table of tickers: NVDA, TSM, AMD, INTC, AVGO, QCOM, ASML, MU, WDC, STX, ARM, SNPS, CDNS
% - Rationale for choosing the semiconductor sector
% - S&P 500 (^GSPC) as market reference

\subsection{Data Source and Period}
% - Yahoo Finance via yfinance library
% - Time period: [specify exact dates]
% - Frequency: daily OHLCV data
% - Any data cleaning steps (missing values, corporate actions)

\subsection{Descriptive Statistics}
% - Summary statistics table (mean return, volatility, correlation with SPX)
% - Correlation heatmap of the asset universe

% ============================================================
% 4. FEATURE ENGINEERING
% ============================================================
\section{Feature Engineering}

% TODO: Write feature engineering section (~1.5-2 pages)
% This is a key section — the features drive the clustering

\subsection{Overview}
% - Features designed to capture market regime, momentum, and relative behaviour
% - Multi-index DataFrame structure: (Datetime, Ticker) -> Feature columns

\subsection{Momentum Features}
% - Simple returns
% - Rate of Change (ROC)
% - MACD (Moving Average Convergence Divergence)
% - RSI (Relative Strength Index)
% - Include formulas for each

\subsection{Volatility Features}
% - Historical volatility (rolling standard deviation of returns)
% - ATR (Average True Range)
% - Include formulas

\subsection{Correlation Features}
% - Rolling correlation with S&P 500 (market beta proxy)
% - Pairwise rolling correlations between stocks

\subsection{Trend Features}
% - Distance from moving averages (20-day, 50-day)
% - Price relative to recent highs/lows

\subsection{Technical Indicators}
% - Bollinger Band position (% B)
% - Stochastic oscillator
% - Williams \%R
% - Brief description of each and why included

\subsection{Feature Normalisation}
% - Standardisation approach (z-score normalisation per feature per timestamp)
% - Handling of NaN values from rolling windows

% ============================================================
% 5. METHODOLOGY
% ============================================================
\section{Methodology}

% TODO: Write methodology section (~3-4 pages — largest section)

\subsection{Clustering Framework}
% - General approach: apply clustering at each timestamp independently
% - Input: feature vector per stock at time T
% - Output: cluster labels per stock at time T
% - Noise label (-1) for stocks not assigned to any cluster (OPTICS, DBScan)

\subsection{OPTICS}
% - Algorithm description
% - Reachability distance and ordering
% - Xi method for cluster extraction
% - Parameters: min_samples=3, xi=0.05, min_cluster_size=3
% - Advantages: variable-density clusters, no need to pre-specify k
% - Include pseudocode or algorithm block

\subsection{K-Means}
% - Algorithm description
% - Centroid-based, iterative assignment and update
% - Choosing k: elbow method and silhouette analysis
% - Parameters: [specify chosen k or selection method]
% - Key difference: assigns every point to a cluster (no noise label)
% - Limitations: assumes spherical clusters, requires specifying k
% - Include pseudocode or algorithm block

\subsection{DBScan}
% - Algorithm description
% - Eps-neighbourhood and core points
% - Parameters: eps (tuned via k-distance graph), min_samples
% - Produces noise labels like OPTICS
% - Key difference from OPTICS: fixed density threshold
% - Include pseudocode or algorithm block

\subsection{Formation and Dissolution Detection}
% - Definition of a formation event: pair transitions from different clusters to same cluster
% - Definition of a dissolution event: pair transitions from same cluster to different clusters
% - Tracking logic across timestamps
% - Include diagram showing formation/dissolution timeline

\subsection{Pair Validation}
% - Cluster persistence: proportion of time pair remains co-clustered over window H
% - Spread computation: beta-hedged spread = Price(A) - beta * Price(B)
% - AR(1) coefficient (phi): mean reversion speed
% - Bounce rate: proportion of spread excursions reverting >= 30\% within 2 periods
% - Composite score for filtering tradeable pairs
% - Include formulas for each metric

\subsection{Correctness Checks}
% - Random baseline permutation test
% - Out-of-sample train/test split
% - OPTICS sensitivity analysis (varying xi)

\subsection{Factor Attribution}
% - Z-score comparison: feature values at formation times vs baseline
% - Formula: Z = (feature_at_formation - mean_baseline) / std_baseline
% - Identifies which features precede formations

\subsection{Predictive Modelling}
% - Classification task: predict formation at time T using features at T-1
% - Class imbalance handling (~2-5\% positive rate)
% - Time-series cross-validation (expanding window, no look-ahead bias)
% - Models: Random Forest Classifier, Gradient Boosting Classifier
% - Hyperparameters: [specify]
% - Evaluation metrics: AUC, precision, recall, accuracy

% ============================================================
% 6. RESULTS
% ============================================================
\section{Results}

% TODO: Write results section (~2-3 pages)

\subsection{OPTICS Clustering Results}
% - Number of clusters detected per timestamp (distribution)
% - Number of formation/dissolution events
% - Random baseline comparison (real vs shuffled formation rates)
% - Sensitivity analysis results (varying xi)
% - Include: cluster distribution plot, formation timeline

\subsection{K-Means Clustering Results}
% - Optimal k selection (elbow plot, silhouette scores)
% - Number of formations detected
% - Comparison to OPTICS on same data
% - Include: elbow plot, silhouette plot

\subsection{DBScan Clustering Results}
% - Eps selection (k-distance graph)
% - Number of formations detected
% - Noise point proportion
% - Comparison to OPTICS on same data
% - Include: k-distance plot

\subsection{Comparative Analysis}
% - Side-by-side table: OPTICS vs K-Means vs DBScan
% - Metrics: number of clusters, formations, tradeable pairs, cluster persistence
% - Strengths and weaknesses of each method on this dataset
% - Include: comparison table, bar charts

\subsection{Pair Validation Results}
% - Distribution of cluster persistence scores
% - Distribution of bounce rates
% - Top tradeable pairs and their metrics
% - Include: validation metrics table for top pairs

\subsection{Factor Attribution Results}
% - Z-score bar chart: which features are most unusual at formation times
% - Feature distribution overlays (baseline vs formation)
% - Key findings: e.g., RSI and correlation are strongest predictors

\subsection{Prediction Results}
% - Model performance comparison (Random Forest vs Gradient Boosting)
% - ROC curves with AUC scores
% - Precision-recall curves
% - Feature importance rankings
% - Cross-validation fold-by-fold results
% - Include: ROC plot, precision-recall plot, feature importance bar chart

% ============================================================
% 7. DISCUSSION
% ============================================================
\section{Discussion}

% TODO: Write discussion (~1-1.5 pages)

\subsection{Interpretation of Results}
% - Which clustering method best captures transient correlations and why
% - Why OPTICS may outperform (variable density, no k specification)
% - Why K-Means may struggle (spherical assumption, forced assignment)
% - Where DBScan fits (fixed density threshold trade-off)

\subsection{Practical Implications}
% - How a trader could use this framework
% - Signal generation: formation detection -> spread monitoring -> entry/exit
% - Risk considerations: bounce rate as confidence measure

\subsection{Limitations}
% - Transaction costs not modelled
% - Semiconductor sector only (generalisability unknown)
% - Formation prediction AUC modest (~0.60-0.75)
% - Feature engineering choices may introduce bias
% - Daily frequency limits applicability to intraday trading

% ============================================================
% 8. CONCLUSION AND FUTURE WORK
% ============================================================
\section{Conclusion and Future Work}

% TODO: Write conclusion (~0.5-1 page)

\subsection{Summary}
% - Recap: what we did, what we found, which method works best

\subsection{Future Work}
% - Real-time deployment and live testing
% - Multi-timeframe analysis (hourly + daily)
% - Sector expansion beyond semiconductors
% - Additional features: volume, options implied volatility, order flow
% - Trading module: full backtest with transaction costs and slippage
% - Ensemble approach: combine predictions from all three clustering methods

% ============================================================
% REFERENCES
% ============================================================
\newpage
\bibliographystyle{plainnat}
% \bibliography{references}  % Uncomment when references.bib is created

% Placeholder references — replace with .bib file
\begin{thebibliography}{99}

\bibitem{gatev2006}
Gatev, E., Goetzmann, W.N. and Rouwenhorst, K.G., 2006.
Pairs trading: Performance of a relative-value arbitrage rule.
\textit{The Review of Financial Studies}, 19(3), pp.797--827.

\bibitem{vidyamurthy2004}
Vidyamurthy, G., 2004.
\textit{Pairs Trading: Quantitative Methods and Analysis}.
John Wiley \& Sons.

\bibitem{ankerst1999}
Ankerst, M., Breunig, M.M., Kriegel, H.P. and Sander, J., 1999.
OPTICS: Ordering points to identify the clustering structure.
\textit{ACM SIGMOD Record}, 28(2), pp.49--60.

\bibitem{ester1996}
Ester, M., Kriegel, H.P., Sander, J. and Xu, X., 1996.
A density-based algorithm for discovering clusters in large spatial databases with noise.
\textit{Proceedings of the 2nd International Conference on Knowledge Discovery and Data Mining (KDD-96)}, pp.226--231.

\bibitem{macqueen1967}
MacQueen, J., 1967.
Some methods for classification and analysis of multivariate observations.
\textit{Proceedings of the 5th Berkeley Symposium on Mathematical Statistics and Probability}, 1, pp.281--297.

% TODO: Add more references as needed

\end{thebibliography}

% ============================================================
% APPENDIX
% ============================================================
\newpage
\appendix

\section{Full Feature List}
% TODO: Table listing every feature, its formula, and its category

\section{Additional Plots}
% TODO: Any supplementary figures not in the main body

\section{Code Repository}
% TODO: Link to GitHub repo or brief description of code structure

\end{document}
